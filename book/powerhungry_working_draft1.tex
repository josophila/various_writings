\documentclass[12pt]{book}
\author{Joseph Cammarata}
\title{Power Hungry}
\date{2015}



\begin{document}
\maketitle

\chapter{Emorob}
\section{Entities of Power}

``In the dark, an object.  Picture it; blackness not broken for thousands of years, but there, an object.  It is not waiting; its power knows no time.  It is not hoping; it does not care.  There, in the blackness, the source of infinite power.''

``How did it arrive?'' She asked.

``Put, by an Emorob.''

``Where did it come from?''

``I do not know.''

Astil sat, eyes fixed on the floor, lost in thoughts too deep to grasp with her conscious mind. Then she asked, ``Is this the secret kept from the young?''

He answered, ``It is one.''

``Tell me the others.'' 

``The other secrets are for the old.  Would you like to be made old?'' He asked, looking down at her from his chair.

For a moment, she felt defiant, but with a deliberate breath she composed herself.  ``I am sorry.  That such a source of power exists unsettles me, and when I imagine what we could do with such an object, I...  want to know what else you know.''

He said nothing.

Astil knew that Yscho had reasons for everything he did.  She used to resent him for how little that was, until one day, similar to this one, he told her how old he was, and how every action -- every motion -- reduced his time left.  There had been a curse, he said. 

She contemplated what she had been told, including Yscho's comment about making her old (could he?).  It had also been the most she had ever heard him speak in a single sitting. She stood up and left Yscho's cave.  Lost in thought, she was unprepared for the blinding rays of the sun, and she recoiled.  As her eyes adjusted to the light, so did her mood.  Her ruminations did not belong in this world of blue sky and glaring sun reflecting off the sand.  They belonged to other worlds: to the cool chasms where water jugs were kept and where you could sit; to Yscho's unchanging cave; to secret places shrouded in unadulterated blackness that housed objects of infinite power.  

With the revitilization of her mood came an excess of energy, and her mind turned to training.  She walked down the hill from Yscho's cave, through desert scrub, to the edge of the  village where she and the rest of the Emorob people lived.  She skirted the village until she arrived at the training center and entered, hoping to find someone to spar with. 

Tril was inside, stretching.  Astil waved from across the large central arena and gestured inquisitively toward one of the sparring rings off to the side.  Tril returned the wave, smiled, and shook her head, returning to her stretches.  

The Emorob people reproduced as a part of a ceremony that took place every ten years.  Yscho designed the mating scheme, and people took part twice during their life: once at age 20 and once again at 30.  Consequently, most children, who were raised communally, were born in distinct generations (called clutches) and always had companions.  Tril was an exception to this; her mother had left the Emorob village for several years - not an uncommon thing for an Emorob to do - and had returned eight months pregnant.  She died after giving birth.

Tril, who at 22 was between generations (Astil was 18).  People were just as kind and welcoming to her as they were to anyone else in the village, since she was still Emorob, but Tril lacked the close friendships that the other villagers could find within their cletch.  As far as anybody knew, the only other resident of the village with no cletch was Yscho, who had always taken a special interest in Tril.

Astil resigned herself to training at the combat dummy in the corner of the arena.  The dummy stood six feet tall and was made of disks stacked on top of one another, connected by a series of springs such that after one disc was twisted in one direction, other disks would rotate in response.  Each disk possessed points around its perimeter where various attachments could be fixed.  The attachments were hard instruments of various shapes and kinematic properties -- some had swinging metal weights, others would spin at high speeds, and others still were contraptions so ingeniously designed as to move almost completely randomly, making it nearly impossible to predict their movements.  Once set into motion, the dummy became a formidable cyclone of pieces to be hit, blocked, and dodged.

Astil felt pleased as she approached the dummy, and thought that Tril, an outlier also in terms of her physical ability, must have been the previous one to use it.  Nearly every attachment point on the disks was occupied, and many with difficult, swinging obstacles.  Astil embraced the challange, hoping to shake off the last clinging vestiges of her meeting with Yscho.

She kicked a peg on the lowest disk to start and the machine sprang into action.  The top disk spun, whipping an iron chain with surprising speed toward her.  Astil ducked, and as the chain spun overhead, she blocked two metal rods that were approaching her at waist and hip height from either side.  After repelling them, the chain swung back towards her face, and as she duck again a long, low pole swept her feet out from under her and sent her twisting toward the floor.  She barely managed to break her fall. 
    
Astil stood up and stepped back, appriasing the dummy.  She readied herself, stepped forward, and this time set the machine into action by shoving against a knee-high metal prong.  The low pole whistled toward her feet, and she repelled it with her foot.  A metal pendulum swung toward her at shoulder height; she caught the weight and forcefully sent it back.  Immediately, a spinning metal pitchfork and a weighted club closed in from both sides at hip and rib level -- Astil barely managed to block these with a knee and forearm but was knocked tumbling backwards by a blow to the head from the iron chain.  

Astil sat in a daze until a shadow appeared on the floor next to her.  Looking first at the mysterious shadow, she then recovered her wits and traced the shadow back to the one who cast it -- Tril, who stood several feet behind her, smiling.  

``Would you like to know what you did wrong?'' asked Tril.

``During the first round or the second round?'' 

``Well, during your first try you made two mistakes, both obvious.''  

Astil nodded solemnly.  ``I realized almost immediately: I failed to appraise the dummy well enough before I began.  Essentially, I was locked into a losing position the moment I kicked the low peg.''

``And the second mistake you made?''  

Astil thought for a moment, but could only shake her head.

``Your second mistake was having the very thought you just described.  In believing your position unwinnable, it was only a matter of time until you made a careless mistake.  Granted, it was a smaller amount of time than I expected, and your mistake was especially careless.''

Before Astil could reply, Tril stepped to the training dummy and kicked the lower peg hard.  The chain immediately swung at her face, but rather than ducking, Tril grabbed the chain with both hands and jerked hard, reversing the direction of the chain and the two objects coming for her midsection.  Then, a knee-high metal plane, like the head of a shovel, whirled towards her.  In a single deft movement, Tril used the chain to pull herself over the shovel head, and as she landed on the other side, kicked the shovel head in the direction it was moving, giving it even more momentum.  In response, a wooden club hurtled toward Tril's chest, at a speed Astil had never seen the dummy acquire before.  With a terrifying facial expression - a snarl wrapped around her lips, brows crashing downward in an angular attack on her nose, and eyes unbelievably piercing, open so wide as to take in the world in an instant - Tril struck the club with the heel of her palm and shattered it into dozens of splintered fragments and instantly changed into roundhouse kick with which she dislodged the top disk, sending it clamouring across the room.  

 The dummy, with springs and wires sticking from its top, stood still with its mechanism of action broken.  Astil watched as Tril's breathing slowed and as her facial expression changed from the terrifying one she had just witnessed, to one of seriousness and almost of suffering, and then quickly, almost imperceptibly, to the confident smirk that Tril normally wore.  Astil was unsettled and in disbelief of her own eyes; that a single person could wear three such faces at all, never mind in so short a time, seemed impossible.  Tril turned and looked directly at Astil, and for a moment Astil's heart jumped with fear as she thought she saw, or worried she would see, that terrible face again.  
 
 Instead Tril, with expressionless eyes leveled at Astil, said ``Last piece of advice: never think of this thing as a dummy.  When you are fighting, you are fighting your enemy, and should be ready to kill at any moment.  Now, clean this up.''

Tril turned and started to walk toward the exit, but Astil stopped her, asking ``What about my second round? If my mistakes during my first round were obvious, what about the second?''
 
Tilting her head back, Tril said ``Maybe next time \emph{emoruquom}.''  Then she walked cooly out the door.

``\emph{Emoruquom},'' thought Astil, and she grimaced.  It was an Emorob word used by elder females to refer to those two generations younger than them -- to those with whom one will never overlap with during the reproduction ceremony.  The word was loaded with many feelings; often with the compassion held by the elderly for the young that they had helped raise, but it always emphasized a divide, an otherness.  Using it, Tril set herself aside from Astil, and being that they were near the same age and were expected to share a close bond, even if they were not of the same cletch, Tril had also set herself aside from the Emorob way of life.

\section{Water and Wishes}


\chapter{Sand in the Wind} %Astil leaves the village in order to learn the cultures of the present day and the history of the world that existed before the Emorob hid the object of power.  A short time into her travels, she is spotted by Teratum, who is also searching for clues of the past and who recognizes her as an Emorob.  She meets Quent (Proo), and at first avoids him and his attempts to befriend her.  Slowly, their friendship grows over time as they discover a genuine affinity for one another; Astil is confident but compassionate, and recognizes Quent's playful/flamboyant exterior as a cover for a damaged, ailing self. They travel together, and their bond grows.  After they learn much about the source of power and the world's past, Tril catches up with them. 

\chapter{Seed Dispersal in the Desert} %The training received by Tril from Yscho, and her quest to find her father.  Tril is full of mixed emotions about what she would finally do when she found him.  Time and time agian, she tries to confront or capture him, but he always avoids her.  However, she does piece together what he is after, and that he has found another Emorob to try to learn from/follow.  Tril does not know at first that this is Astil, but soon figures it out after encountering Astil and Quent.  She tries to convince Astil to cease her quest, and to be wary of Quent, leading to combat, which Tril wins without much difficulty.  After, she refuses to tell Astil more. 

\chapter{The Life of Proo} %Known as Quent later in his life, the horribly early years of Proo, and the experiences that ultimately lead to his death. Covering also his betrayal of Astil, but his lasting reluctance to blame Teratum for the negative occurances that Teratum arguably caused.  Also includes Proo's addiction to euphoria-inducing drugs and then to sex slaves.  Teratum rescues him from this life, earning Proo's loyalty, but Proo's self-esteem never improves while he is friends with Teratum.  He is initially befriended by Teratum, who is looking for some companionship but also for someone to run errands/do work.  Eventually he convinces Proo to try to befriend or, if possible, make Astil fall in love with him.  

%many more things happen, but in the end, Astil finds out that Proo was sent to her by Teratum, and their relationship falls apart.  Proo seeks solace first in sex, then in drugs, and then finally commits suicide. 



\chapter{The Art of Sloughing} %Teratum's life before meeting Proo; Marriage, then meeting Tril's mother, then learning of the object of power and his subsequent determination to obtain it.

\chapter{Pieces of Broken Lives} %Teratum could have stopped Proo from committing suicide, but does not.  He later regrets this moment immenseley, and it leads to him  contemplating his behavior from the rest of his time, and he eventually seeks redemption, although he struggles with it. Giving up on his ambition, he lets his guard down and is immediately captured by Tril, who is confused by the man in front of her, as she had expected more or less what he used to be.  After much contemplation, she brings him back to Yscho, who hears Teratum's confession, then tells Teratum where to find the object of power.

\chapter{}

\end{document}